\chapter{Introduction}
\label{chap:introduction}

% Chapter overview: Introduce the problem of blood disease detection, motivate the need for automated systems, state the problem clearly, present our proposed solution, highlight novel contributions, and outline the thesis structure.

\section{Motivation}
\label{sec:motivation}

% Blood diseases (malaria, anemia, leukemia) affect millions globally
% Traditional diagnosis: manual microscopy - time-consuming, expertise-dependent
% Need for automated, accurate, scalable diagnostic tools
% AI/ML offers promise but faces challenges in medical imaging


\section{Problem Statement}
\label{sec:problem_statement}

% CHALLENGE 1: Localization
% - Raw blood smear images contain multiple cells
% - Cannot identify WHICH cells are infected
% - Cannot COUNT number of infected cells

% CHALLENGE 2: Data Scarcity
% - Most datasets: raw microscopy images (not single-cell labeled)
% - Training classifiers requires single-cell annotated data
% - Limited availability of properly segmented datasets

% CHALLENGE 3: Orientation Invariance
% - Blood cells have no inherent orientation
% - Same cell at different rotations should yield same diagnosis
% - Traditional CNNs require extensive data augmentation


\section{Proposed Solution}
\label{sec:proposed_solution}

% A three-stage hybrid pipeline:

\subsection{Stage 1: Segmentation}
\label{subsec:stage_segmentation}

% - Pre-trained Mask R-CNN fine-tuned for blood cell detection
% - Isolates individual RBCs and WBCs from raw smears
% - Generates synthetic single-cell labeled dataset


\subsection{Stage 2: Classification}
\label{subsec:stage_classification}

% - RBC Classification: Reynolds Network (ReyNet) architecture
%   * Exploits rotational/flip symmetry mathematically
%   * Reduces parameter redundancy
%   * Improves computational efficiency
% - WBC Classification: Fine-tuned ResNet18
%   * Leverages pre-trained features
%   * Transfer learning for limited data


\subsection{Stage 3: Disease Detection \& Counting}
\label{subsec:stage_detection}

% - Aggregates cell-level predictions
% - Counts infected vs healthy cells
% - Provides diagnostic report with spatial localization


\section{Novel Contributions}
\label{sec:contributions}

% 1. First application of Reynolds Networks to blood cell classification
% 2. Hybrid pipeline addressing localization, classification, and counting
% 3. Synthetic dataset generation using segmentation for classifier training
% 4. Comparative analysis: ReyNet vs traditional CNNs for medical imaging


\section{Thesis Organization}
\label{sec:organization}

% Chapter 2: Literature review on blood disease detection, CNNs, and equivariant networks
% Chapter 3: Detailed methodology of pipeline components
% Chapter 4: Experimental results and comparative analysis
% Chapter 5: Conclusions and future research directions
