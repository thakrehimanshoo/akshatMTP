\chapter{Literature Review}
\label{chap:literature_review}

% Chapter overview: Survey existing work in blood disease detection, convolutional neural networks, segmentation models, and symmetry-based neural networks. Identify the research gap this thesis addresses.

\section{Blood Disease Detection}
\label{sec:blood_disease_detection}

\subsection{Traditional Methods}
\label{subsec:traditional_methods}

% - Manual microscopy: gold standard but labor-intensive
% - Flow cytometry: expensive, requires specialized equipment
% - Limitations: inter-observer variability, scalability issues


\subsection{Machine Learning Approaches}
\label{subsec:ml_approaches}

% - Early methods: handcrafted features + SVM/Random Forest
% - Deep learning revolution: CNNs for medical image analysis
% - Challenges: limited labeled data, class imbalance


\section{Convolutional Neural Networks}
\label{sec:cnns}

\subsection{Standard Architectures}
\label{subsec:standard_architectures}

% - AlexNet, VGG, ResNet: pioneering architectures
% - Skip connections (ResNet): address vanishing gradients
% - Transfer learning: pre-trained models on ImageNet


\subsection{Medical Imaging Applications}
\label{subsec:medical_imaging_applications}

% - Skin lesion classification, retinal disease detection
% - X-ray analysis, pathology image classification
% - Success factors: large datasets, data augmentation


\section{Segmentation Models}
\label{sec:segmentation_models}

\subsection{Instance Segmentation}
\label{subsec:instance_segmentation}

% - Mask R-CNN: extension of Faster R-CNN with mask branch
% - Architecture: backbone + RPN + RoI Align + mask head
% - Applications: object detection + pixel-level segmentation


\subsection{Blood Cell Segmentation}
\label{subsec:blood_cell_segmentation}

% - Challenges: overlapping cells, staining variations
% - Existing approaches: U-Net, FCN, Mask R-CNN variants
% - Fine-tuning strategies for limited medical data


\section{Symmetry and Equivariance in Neural Networks}
\label{sec:symmetry_equivariance}

\subsection{Group Equivariant CNNs}
\label{subsec:group_equivariant_cnns}

% - Conventional CNNs: translation equivariant
% - Need for rotation equivariance in medical imaging
% - Group convolutions: G-CNNs (Cohen & Welling, 2016)


\subsection{Reynolds Networks}
\label{subsec:reynolds_networks}

\subsubsection{Reynolds Operator Theory}
\label{subsubsec:reynolds_theory}

% - Equivariant operator: τ_G(f)(x) = (1/|G|) Σ_{g∈G} g^{-1}·f(g·x)
% - Invariant operator: γ_G(f)(x) = (1/|G|) Σ_{g∈G} f(g·x)
% - Converts any function to equivariant/invariant form


\subsubsection{Key Innovations}
\label{subsubsec:reynolds_innovations}

% - Reynolds Design: subset H ⊂ G where averaging over H ≈ averaging over G
% - Reduces computational complexity: O(n!) → O(n²) for graphs
% - Reynolds Dimension: minimum input dimension needed for universality
% - Reduces parameter redundancy


\subsubsection{Theoretical Guarantees}
\label{subsubsec:reynolds_guarantees}

% - Theorem 12 (Universality): ReyNets are universal approximators
% - Theorem 3 (Representation): Equivariant maps decompose via basis tableaux
% - Reduces network complexity while maintaining expressiveness


\subsubsection{Applications}
\label{subsubsec:reynolds_applications}

% - Graph neural networks, point clouds, molecular property prediction
% - NOT yet applied to medical imaging or blood cell analysis


\subsection{Why ReyNet for Blood Cells?}
\label{subsec:reynet_for_blood_cells}

% - Blood cells exhibit rotational symmetry (no preferred orientation)
% - Traditional approach: data augmentation (rotate/flip images)
%   * Increases dataset size artificially
%   * Longer training time
%   * Network doesn't learn symmetry inherently
% - ReyNet approach: build symmetry into architecture
%   * Network learns rotation-invariant features natively
%   * More parameter-efficient
%   * Theoretically grounded in group theory


\section{Research Gap}
\label{sec:research_gap}

% EXISTING WORK:
% - Blood disease detection: mostly standard CNNs
% - Reynolds Networks: applied to graphs/point clouds, NOT medical images
% - Medical segmentation: well-established (Mask R-CNN)
% - Equivariant networks: theoretical but limited medical applications

% THIS THESIS:
% - First to apply ReyNet to blood cell classification
% - Combines segmentation + classification in unified pipeline
% - Addresses localization, classification, AND counting
% - Bridges gap between theoretical equivariance and practical medical AI
