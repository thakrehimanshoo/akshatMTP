\chapter{Methodology}
\label{chap:methodology}

% Chapter overview: Detail the proposed hybrid pipeline including dataset preparation, Mask R-CNN segmentation, Reynolds Network architecture, WBC classification, pipeline integration, and evaluation metrics.

\section{Dataset and Preprocessing}
\label{sec:dataset_preprocessing}

\subsection{Dataset Description}
\label{subsec:dataset_description}

% - Source: [Specify your dataset source]
% - Total images: [Number]
% - Cell types: RBCs (healthy/infected), WBCs (types)
% - Disease categories: [List diseases - malaria, anemia, etc.]


\subsection{Data Challenges}
\label{subsec:data_challenges}

% - Raw blood smear images (not single-cell)
% - Class imbalance (more healthy than infected)
% - Staining variations across images
% - Overlapping cells in dense regions


\subsection{Preprocessing Pipeline}
\label{subsec:preprocessing_pipeline}

% 1. Image normalization: rescale to [0, 1]
% 2. Stain normalization: [if applicable]
% 3. Image resizing: 130×130 for ReyNet input
% 4. No manual rotation augmentation (ReyNet handles this)


\section{Segmentation Pipeline: Mask R-CNN}
\label{sec:segmentation_maskrcnn}

\subsection{Architecture Overview}
\label{subsec:maskrcnn_architecture}

% - Backbone: ResNet-50 with Feature Pyramid Network (FPN)
% - Region Proposal Network (RPN): generates candidate regions
% - RoI Align: extracts features for each proposal
% - Mask Head: produces pixel-level segmentation masks


\subsection{Fine-tuning Strategy}
\label{subsec:maskrcnn_finetuning}

% - Pre-trained weights: COCO dataset
% - Transfer learning approach:
%   * Freeze early layers (general features)
%   * Fine-tune later layers + mask head
% - Loss function: L_total = L_cls + L_box + L_mask


\subsection{Training Configuration}
\label{subsec:maskrcnn_training}

% - Optimizer: SGD with momentum (0.9)
% - Learning rate: 0.001 with step decay
% - Batch size: [Specify]
% - Epochs: [Specify]
% - Data augmentation: horizontal/vertical flips, brightness adjustment


\subsection{Output}
\label{subsec:maskrcnn_output}

% - Bounding boxes for each cell
% - Binary masks (cell vs background)
% - Confidence scores
% - Generates synthetic single-cell dataset for classification


\section{Reynolds Network for RBC Classification}
\label{sec:reynet_rbc}

\subsection{Reynolds Symmetry Theory}
\label{subsec:reynolds_symmetry_theory}

\subsubsection{Mathematical Foundation}
\label{subsubsec:math_foundation}

% A function f: ℝ^{N×N} → ℝ^{M} is:
% - Equivariant if: f(g·x) = g·f(x) for all g ∈ G
% - Invariant if: f(g·x) = f(x) for all g ∈ G
%
% Reynolds Operator (equivariant):
% τ_G(f)(x) = (1/|G|) Σ_{g∈G} g^{-1}·f(g·x)
%
% For blood cell images:
% - G = D_4 (dihedral group): 4 rotations × 2 flips = 8 transformations
% - x = input cell image (130×130×3)
% - f = pre-model (standard neural network)
% - τ_G(f) = Reynolds model (equivariant by construction)


\subsubsection{Reynolds Design}
\label{subsubsec:reynolds_design}

% Instead of averaging over all g ∈ G (expensive):
% - Use subset H ⊂ G where τ_G(f) ≈ τ_H(f)
% - For graphs: H has order n² instead of n!
% - For images: carefully chosen cyclic subgroups


\subsubsection{Reynolds Dimension}
\label{subsubsec:reynolds_dimension}

% Minimum input variables needed to represent invariant functions
% - Full input: all pixel values
% - Reduced input: d critical dimensions
% - For our case: 4-reduced ReyNet (d=4)
% - Theorem 17 guarantees universality with reduced dimensions


\subsection{ReyNet Architecture}
\label{subsec:reynet_architecture}

\subsubsection{Hybrid ResNet-ReyNet Model}
\label{subsubsec:hybrid_model}

% Input: 130×130×3 RGB image of single RBC
%
% EARLY LAYERS (Standard ResNet):
% - Initial Conv: 64 filters, 7×7, stride 2
% - BatchNorm + ReLU + MaxPool
% - ResNet Block 1-2: standard convolution blocks
%
% MID LAYERS (Standard ResNet):
% - ResNet Block 3: conv_block(256, stride=2) × 2
%
% DEEP LAYERS (ReyNet with Reynolds Symmetry):
% - Reynolds Symmetry Layer
% - ReyNet Convolution Block
% - Final Conv: 512 filters, 3×3
%
% OUTPUT HEAD:
% - GlobalAveragePooling2D
% - Dense(num_classes=2, activation='softmax')


\subsubsection{Reynolds Symmetry Layer}
\label{subsubsec:reynolds_symmetry_layer}

% Applies rotations: 0°, 90°, 180°, 270°
% Applies flip: vertical flip
% Averages: (x + rot90 + rot180 + rot270 + flip_ud) / 5


\subsubsection{ReyNet Convolution Block}
\label{subsubsec:reynet_conv_block}

% - Feature-wise MLP (pointwise convolution)
% - Set-wise MLP (Reynolds operator on graph structure)
% - Implements separated architecture from paper


\subsection{Reynolds Design Implementation}
\label{subsec:reynolds_implementation}

\subsubsection{Mapping 1: Extract Invariant Features}
\label{subsubsec:mapping1}

% - Diagonal elements: x_ii (self-loops)
% - Off-diagonal: x_ij, x_ji (edge connections)
% - Transpose: X^T
% - Stack: [X, X^T, row_diag, col_diag] → 4 dimensions


\subsubsection{Neural Network N: Process d-reduced Input}
\label{subsubsec:neural_network_n}

% - Input: (Batch, N, N, Channels, 4)
% - MLP with hidden units: [32]
% - Output: (Batch, N, N, Channels)


\subsubsection{Mapping 2: Restore Dimensions}
\label{subsubsec:mapping2}

% - Identity (features already in correct space)


\subsubsection{Efficiency Gains}
\label{subsubsec:efficiency_gains}

% - Standard Reynolds operator: O(|G|) = O(8) for D_4 group
% - With Reynolds design: O(n²) for graph interpretation
% - With 4-reduced dimensions: 1/4 parameter reduction


\subsection{Training Configuration}
\label{subsec:reynet_training}

% HYPERPARAMETERS:
% - Optimizer: Adam
% - Learning rate: 1e-4
% - Batch size: 16
% - Epochs: 15
% - Early stopping: patience=3 on validation loss
%
% LOSS FUNCTION:
% - Categorical cross-entropy
% - Corner MSE loss for equivariant tasks
%
% DATA STRATEGY:
% - Input: Single-cell images from Mask R-CNN
% - No manual rotation augmentation
% - Reynolds augmentation: generates symmetry variants internally


\section{WBC Classification: ResNet18}
\label{sec:wbc_classification}

\subsection{Architecture}
\label{subsec:wbc_architecture}

% - Pre-trained ResNet18 on ImageNet
% - Fine-tuning approach:
%   * Freeze early layers (conv1 - layer2)
%   * Unfreeze layer3, layer4, and FC layer
% - Output: [Number of WBC classes]


\subsection{Rationale}
\label{subsec:wbc_rationale}

% - WBC morphology is more complex than RBC
% - Pre-trained features transfer well
% - Smaller WBC dataset → transfer learning beneficial
% - WBC orientation less critical than RBC (nucleus has structure)


\subsection{Training Configuration}
\label{subsec:wbc_training}

% - Optimizer: Adam, lr=1e-4
% - Batch size: 32
% - Epochs: 20
% - Data augmentation: rotation, flip, color jitter


\section{Pipeline Integration}
\label{sec:pipeline_integration}

\subsection{End-to-End Workflow}
\label{subsec:end_to_end_workflow}

% INPUT: Raw blood smear microscopy image
%
% STEP 1: SEGMENTATION (Mask R-CNN)
% → Detect all cells
% → Generate bounding boxes and masks
% → Extract individual cell images
%
% STEP 2: CLASSIFICATION
% RBC path: → ReyNet classifier
% WBC path: → ResNet18 classifier
%
% STEP 3: AGGREGATION
% → Count total cells, infected cells
% → Calculate infection rate
% → Spatial localization
%
% OUTPUT:
% - Disease diagnosis
% - Cell count statistics
% - Annotated image with bounding boxes and labels


\subsection{Integration Challenges}
\label{subsec:integration_challenges}

% - Coordinate mapping: segmentation → classification → visualization
% - Confidence thresholding: filter low-confidence detections
% - Post-processing: non-maximum suppression for overlapping cells


\section{Evaluation Metrics}
\label{sec:evaluation_metrics}

\subsection{Segmentation Metrics}
\label{subsec:segmentation_metrics}

% - Mean Average Precision (mAP) @ IoU thresholds
% - Precision, Recall, F1-score for cell detection


\subsection{Classification Metrics}
\label{subsec:classification_metrics}

% - Accuracy: correct predictions / total predictions
% - Precision: TP / (TP + FP) per class
% - Recall: TP / (TP + FN) per class
% - F1-score: harmonic mean of precision and recall
% - Confusion matrix


\subsection{Rotation Invariance Testing}
\label{subsec:rotation_invariance}

% - Test set augmentation: rotate images at multiple angles
% - Measure prediction consistency
% - Compare ReyNet vs ResNet on rotated inputs


\subsection{Computational Efficiency}
\label{subsec:computational_efficiency}

% - Inference time per image (ms)
% - Parameters count
% - FLOPs (floating point operations)


\subsection{Pipeline Metrics}
\label{subsec:pipeline_metrics}

% - End-to-end accuracy
% - Cell counting error: |predicted_count - ground_truth|
% - Clinical metrics: sensitivity, specificity
