\chapter{Conclusion and Future Work}
\label{chap:conclusion}

% Chapter overview: Summarize contributions, key findings, acknowledge limitations, and propose future research directions.

\section{Summary of Contributions}
\label{sec:summary_contributions}

% This thesis presented a novel hybrid segmentation-classification pipeline for automated blood disease detection. The key contributions are:
%
% 1. FIRST APPLICATION OF REYNOLDS NETWORKS TO MEDICAL IMAGING
% 2. INTEGRATED PIPELINE ARCHITECTURE
% 3. SYNTHETIC DATASET GENERATION
% 4. COMPREHENSIVE EVALUATION


\section{Key Findings}
\label{sec:key_findings}

% - ReyNet achieves comparable accuracy with ~25% fewer parameters
% - Superior rotation invariance: [X%] consistency vs [Y%] for ResNet
% - End-to-end pipeline achieves [Z%] accuracy
% - Cell counting MAE: [Value]
% - Inference time: [Value] ms per image


\section{Limitations}
\label{sec:limitations}

\subsection{Technical Limitations}
\label{subsec:technical_limitations}

% - Reynolds operator O(n²) complexity
% - Two-stage training increases complexity
% - Pipeline error propagation


\subsection{Dataset Limitations}
\label{subsec:dataset_limitations}

% - Limited to 2D microscopy images
% - Single dataset used
% - Class imbalance


\subsection{Clinical Limitations}
\label{subsec:clinical_limitations}

% - Not validated on clinical ground truth
% - Single disease focus
% - Requires validation by medical professionals


\section{Future Work}
\label{sec:future_work}

\subsection{Architectural Extensions}
\label{subsec:architectural_extensions}

% - 3D ReyNet for volumetric imaging
% - Multi-scale ReyNet
% - Attention mechanisms
% - End-to-end training: joint segmentation-classification


\subsection{Computational Optimization}
\label{subsec:computational_optimization}

% - Model quantization for edge deployment
% - Knowledge distillation
% - Sparse Reynolds operators
% - GPU optimization


\subsection{Clinical Applications}
\label{subsec:clinical_applications}

% - Multi-disease detection
% - Severity grading
% - Real-time inference
% - Explainability (Grad-CAM)


\subsection{Dataset and Validation}
\label{subsec:dataset_validation}

% - Multi-center validation
% - Cross-modality transfer
% - Longitudinal studies
% - Clinical trial comparison


\subsection{Theoretical Extensions}
\label{subsec:theoretical_extensions}

% - Extend Reynolds theory to other medical imaging domains
% - Study optimal Reynolds dimension
% - Investigate hybrid symmetries


\section{Broader Impact}
\label{sec:broader_impact}

% - RESOURCE-CONSTRAINED SETTINGS
% - SCREENING PROGRAMS
% - EDUCATION
% - RESEARCH


\section{Concluding Remarks}
\label{sec:concluding_remarks}

% This thesis demonstrates that incorporating mathematical symmetry principles (Reynolds Networks) into deep learning architectures can yield practical benefits for medical image analysis.
%
% The integration of theoretical insights from group theory with practical medical imaging needs exemplifies the value of interdisciplinary research.
