\thispagestyle{plain}
\begin{center}
    \Large \textbf{\uppercase{Abstract}}
\end{center}

\vspace{3\baselineskip}

\noindent
Blood diseases such as malaria, anemia, and leukemia affect millions globally, yet their diagnosis remains dependent on manual microscopy—a time-consuming and expertise-intensive process. This thesis presents a novel hybrid deep learning pipeline that addresses three critical challenges in automated blood disease detection: precise cell localization, accurate classification under data scarcity, and inherent orientation invariance of blood cells.

\vspace{\baselineskip}

\noindent
The proposed solution integrates a three-stage architecture. First, a fine-tuned Mask R-CNN performs instance segmentation on raw blood smear images, isolating individual red blood cells (RBCs) and white blood cells (WBCs) while generating a synthetic single-cell labeled dataset. Second, a Reynolds Network (ReyNet)—a novel architecture based on group equivariant theory—classifies RBCs by mathematically encoding rotational symmetry into the network structure, eliminating the need for extensive data augmentation. Concurrently, a fine-tuned ResNet18 classifies WBC subtypes via transfer learning. Third, the pipeline aggregates cell-level predictions to provide disease diagnosis, infection rate quantification, and spatial localization of infected cells.

\vspace{\baselineskip}

\noindent
This work makes four key contributions: (1) the first application of Reynolds Networks to medical imaging, demonstrating superior rotation invariance and parameter efficiency compared to conventional CNNs; (2) a unified pipeline addressing localization, classification, and counting in a single framework; (3) a data augmentation strategy through segmentation-driven synthetic dataset generation; and (4) comprehensive empirical validation showing that ReyNet achieves comparable accuracy with approximately 25\% fewer parameters while maintaining robust performance across arbitrary image rotations. The system demonstrates clinical viability for automated blood disease screening, particularly in resource-constrained settings where expert pathologists are scarce.

\vspace{\baselineskip}

\noindent
\textbf{Keywords}: Blood Disease Detection, Reynolds Networks, Group Equivariant Neural Networks, Mask R-CNN, Medical Image Segmentation, Deep Learning, Rotation Invariance, Automated Diagnosis