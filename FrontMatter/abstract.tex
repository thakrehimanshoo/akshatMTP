\thispagestyle{plain}
\begin{center}
    \Large \textbf{\uppercase{Abstract}}
\end{center}

\vspace{3\baselineskip}

\noindent
Blood disease diagnosis relies on manual microscopy, which is time-consuming and expertise-dependent. This thesis presents a novel hybrid deep learning pipeline addressing three critical challenges: cell localization, classification under data scarcity, and orientation invariance.

\vspace{\baselineskip}

\noindent
The proposed three-stage architecture integrates: (1) Mask R-CNN for instance segmentation, isolating individual cells and generating synthetic single-cell datasets; (2) Reynolds Network (ReyNet) for RBC classification, mathematically encoding rotational symmetry via group equivariant theory to eliminate data augmentation needs, alongside ResNet18 for WBC classification; and (3) aggregation of predictions for disease diagnosis and cell counting.

\vspace{\baselineskip}

\noindent
This work presents the first application of Reynolds Networks to medical imaging, achieving comparable accuracy with 25\% fewer parameters while demonstrating superior rotation invariance. The unified pipeline addresses localization, classification, and counting, offering clinical viability for automated screening in resource-constrained settings.

\vspace{\baselineskip}

\noindent
\textbf{Keywords}: Blood Disease Detection, Reynolds Networks, Group Equivariant Neural Networks, Mask R-CNN, Medical Image Segmentation, Deep Learning, Rotation Invariance, Automated Diagnosis